\documentclass[11pt]{report}
\usepackage[utf8]{inputenc}
\usepackage[T1]{fontenc}
\usepackage[british]{babel}
\usepackage{graphicx}
\usepackage[dvipsnames]{xcolor}
\usepackage[sb]{libertine}
\usepackage[scale=0.9]{inconsolata}

\usepackage[sf]{titlesec}
\usepackage[square,sort,colon,authoryear]{natbib}
%\usepackage[export]{adjustbox}

\usepackage{
  marginnote, % Improved margin notes
  environ,
  ragged2e,   % Justified text in the margin notes
  url,        % For typesetting URLs
  listings,   % Code formatting
  lipsum,
  booktabs,
  changepage,
  float,
  dingbat
}
\renewcommand{\bfdefault}{bx}
\DeclareTextFontCommand\textsb{\libertineSB}

\usepackage[hidelinks]{hyperref}
\usepackage[capitalise]{cleveref}
\Crefname{section}{\S}{\S\S}

\usepackage{enumitem}   % Improved lists 
% \setlist{noitemsep} % To make all lists compact
\setlist[itemize]{itemsep=0pt}
\setlist[itemize,1]{label=--}
\setlist[itemize,2]{label=\ensuremath{\triangleright}}
\setlist[itemize,3]{label=\RED{AVOID}}
% Description lists to use semibold labels
\setlist[description]{font=\libertineSB}

\usepackage[twoside,labelfont=sf]{caption}

\captionsetup{justification=raggedright,singlelinecheck=false}

\newcommand\myhrulefill[1]{\leavevmode\leaders\hrule height#1\hfill\kern0pt}
\DeclareCaptionFormat{FigFormat}{{\color{black}\myhrulefill{0.5pt}}\\#1#2#3}
\captionsetup[figure]{format=FigFormat}
\captionsetup[table]{format=FigFormat}

\DeclareCaptionFormat{LstFormat}{\textsf{Listing}~\arabic{chapter}.\arabic{listing}:#2#3}
\floatstyle{ruled}
\newfloat{listing}{thp}{lol}[chapter]
\floatname{listing}{Listing}
\captionsetup[listing]{format=LstFormat}

\NewEnviron{MarginNote}[1][0mm]{\marginnote{\footnotesize\justifying\BODY}[#1]}
\newcommand{\Footnote}[2][0mm]{\footnotemark\marginnote{\footnotesize$^{\arabic{footnote})}$~#2}[#1]}

\renewenvironment{figure*}[1][]{%
  \begin{figure}[#1]%
	% This feature is deprecated for now
	% \checkoddpage%
	% \ifoddpage%
	%   \begin{adjustwidth}{0cm}{-45mm}%%
	% \else%
	%   \begin{adjustwidth}{-45mm}{0cm}%%
	% \fi%
	}{%
	% \end{adjustwidth}%
  \end{figure}}

\renewenvironment{table*}[1][]{%
  \begin{table}[#1]%
	% This feature is deprecated for now
	% \checkoddpage%
	% \ifoddpage%
	%   \begin{adjustwidth}{0cm}{-45mm}%%
	% \else%
	%   \begin{adjustwidth}{-45mm}{0cm}%%
	% \fi%
	}{%
	% \end{adjustwidth}%
  \end{table}}

\renewenvironment{listing*}[1][]{%
  \begin{listing}[#1]%
	% This feature is deprecated for now
	% \checkoddpage%
	% \ifoddpage%
	%   \begin{adjustwidth}{0cm}{-45mm}%%
	% \else%
	%   \begin{adjustwidth}{-45mm}{0cm}%%
	% \fi%
	}{%
	% \end{adjustwidth}%
  \end{listing}}

%% == Code =======================================================
\lstnewenvironment{Code}[1][style=std]{\lstset{#1}}{}
\lstnewenvironment{Code_Numbered}[1][style=std,numbers=left]{\lstset{#1}}{}

\renewcommand{\c}[1]{\lstinline[style=std]@#1@}

\lstdefinestyle{std}{
  language=java,
  basicstyle=\small\tt\color{black},
  keywordstyle=\small\tt\bfseries,
  numberstyle=\footnotesize\sf\color{black},
  commentstyle=\small\color{black}\it,
  aboveskip=1ex,
  belowskip=1ex,
  tabsize=2,
  columns=fullflexible,
  xleftmargin=1ex,
  resetmargins=true,
  showstringspaces=false,
  morecomment=[l]{//},
  morecomment=[l]{--},
  morecomment=[s]{/*}{*/},
  escapeinside=@@,
  morekeywords={Frobies},
  moredelim=[is][\textit]{___}{___},
  moredelim=[is][\textbf]{__*}{*__},
  numberbychapter=true
}

\usepackage[activate={true,nocompatibility},final,tracking=true,kerning=true,spacing=true,factor=1100,stretch=10,shrink=10]{microtype}
\usepackage[paper=a4paper,text={13cm,24cm},marginparsep=5mm,marginparwidth=45mm,inner=20mm,twoside]{geometry}

\newcommand{\RED}[1]{\textcolor{red}{#1}}
\newcommand{\ie}{\emph{i.e.,}}
\newcommand{\eg}{\emph{e.g.,}}
\newcommand{\etal}{\emph{et~al.}}

\usepackage{pifont}
\newcommand{\Yes}{\ding{51}}
\newcommand{\No}{\ding{55}}

\renewcommand{\bfdefault}{b}
\clearpage{\pagestyle{empty}\cleardoublepage}

%% You can remove this line if you compile with --synctex=1 (see Makefile)
\synctex=1
\pagestyle{plain}

\begin{document}

\title{Group 15 Project Report}
\author{Adnan Erlangga Raharja, Ha Pham, Yasiru Bhagya, Yoav Kalisher}
\date{\today}

\maketitle

\vspace*{3cm}
\section*{Abstract}

Today, students in higher education must manage a wide variety of tasks in very constrained timeframes as a part of their studies, which necessitates the use of tools to better manage their workload. Prior research has shown that poor time management and academic stress can significantly affect student performance and well-being \citep{misra2000college}; likewise, effective learning strategies and self-regulation are key predictors of academic success \citep{zimmerman2002becoming}.
However, despite how many of these tools exist, very few of them are tailored for students. Our team hopes to create a free tool that can fill this gap. We aim to create a web-based task management tool that provides students with the ability to manage both their individual work and work they need to do in groups.


\tableofcontents



\chapter{Introduction}

Task management systems are usually used by the project development teams to track their tasks from beginning to end, delegate subtasks to teammates, and set deadlines to make sure projects get done on time \citep{nurzi2022web}. While these systems are traditionally associated with professional project development teams, the same concept is now increasingly applied in educational contexts. Studies highlight that higher education students often face cognitive overload when balancing multiple courses and activities \citep{kirschner2013learners}. 

Today, students in higher education must manage a wide variety of tasks in very constrained timeframes as a part of their studies, from individual homework assignments and readings to group projects and preparation for exams. This workload necessitates tools that enable students to organize and prioritize their work to be successful. Because of how ubiquitous computer technology is today, there are a large number of tools that leverage the internet to allow for easy access from most consumer computer devices. Some of the most common and well-known task management applications, widely used by professionals, are Trello \citep{shchetynina2022trello}, Todoist \citep{bourbon2021time}, and Notion \citep{reach2022notion}, as they combine ease of use with features such as visual organization, reminders, and flexible customization.

However, despite how many of these tools exist, very few of them are tailored for students. Many of them are designed either for large corporate teams or for individuals with very simple needs, neither of which work well for the in-between needs that students have. Task managers that are more fitting for students are often prohibitively expensive for them, as they require yearly subscriptions and most students don’t have a consistent expendable income. As a result, there aren’t many good options for students. This is especially true when it comes to managing group work for classes, something that is particularly relevant when studying at a Swedish university where it is very common to have classes that primarily consist of group work.
Our team hopes to create a free tool that can fill this gap. We aim to create a web-based task management tool that provides students with the ability to manage both their individual work and work they need to do in groups. As a part of making this tool accessible for a large number of users, the app should also be lightweight and able to run on even less computationally-powerful devices such as a simple raspberry pi \citep{upton2016raspberry} so that any student can download the code and run their own instance of the task manager without having to rely on an external service.

\chapter{Background}
\label{cha:background}

A wide variety of internet-based task management tools currently exist. These are offered by a wide variety of companies, some of which specialize in creating these tools while others provide them as a part of a larger suite of tools.
Google, Microsoft, and Apple, companies with a major web presence and billions of active user accounts, all provide simple task managers as a part of their free accounts that are accessible from any device with a web browser. These tend to have limited functionality, allowing for the creation of simple tasks containing a title and a due date. Microsoft’s and Apple’s task managers also allow for collaboration with more than one person, but both require that all users have an account with them. Despite these companies’ wide reach, it is common for individuals to only have active accounts with one or neither of these two companies. In Apple’s case in particular, the company’s Reminders app is only available on Apple’s platforms, further shrinking the availability of the tool.
Other companies, such as Doist, Inc., the Omni Group, and Cultured Code provide more complex and feature-rich applications, but their products all require costly purchases or subscriptions. Furthermore, of these three companies, only Doist offers a task manager that allows for collaboration, but this application is also the most expensive of the three.
Finally, Atlassian and Asana create tools that are specifically designed for working in teams. However, these tools are specially designed for specific workflows in corporate teams, and are inflexible and difficult to use in other settings. Group work in higher education has been widely recognized as both a pedagogical opportunity and a source of coordination challenges \citep{johnson2009educational}; studies emphasize that collaborative learning, while beneficial, requires effective task organization and communication tools \citep{dunlosky2013improving}.

We have also explored the efficacy of leveraging free general-purpose note taking tools, such as Google docs, as task management tools in groups. While this was technically possible, it was unwieldy and difficult to manage, as tasks could easily get lost in large projects and having multiple team members creating and updating tasks at once made the document difficult to track in real time, leading to both redundancy and missing tasks.


\chapter{Methodology}
Students have four key types of tasks that we have identified as a part of their studies: 
Individual homework assignments – These entail work that each student does by him- or herself, often consisting of multiple parts to complete. For example, a student may have to submit a C programming assignment that has to enable a specific set of functionality, or a student may have to write an essay about a topic with a specific structure. These often have strict deadlines, either for individual components or for the submission as a whole.
Self-guided learning – Many classes require students to do additional self-guided learning outside of lectures in the form of articles and books to read or videos to watch. These vary in length, content, and structure, but can usually be broken down into discrete subparts. Most teachers do not have formal deadlines for the learning, but will reference it throughout the course or will expect the student to use the knowledge in various assignments or exams.
Exam preparation – Many courses end with a comprehensive exam covering all of the material taught in the course. Evidence shows that some study techniques are much more effective than others \citep{dunlosky2013improving}, which reinforces the importance of structured preparation tools. Succeeding in these exams requires extensive preparation and review of the course’s provided resources, and may also entail completing practice exams or exercises provided by the instructor.
Group work – Group work is similar to individual work, with the added complexity of having to communicate and orchestrate tasks with other students. This means all members of the group should have easy access to not just which tasks need to be done for an assignment, but also the current state of all tasks and who is assigned to work on them. This also includes needing to be able to schedule times when work needs to be done in teams.
We determined the functional requirements for our task management tool based on our own experiences as students getting master’s degrees from Uppsala University. Of our four group members, two of us are majoring in computer science, and two are majoring in embedded systems. We used our current courses for reference on the kind of tasks we would need to be able to track in the application, and compared them to the tools we currently have access to. Furthermore, we scoped the project based on specific constraints provided by the course. We had 8 weeks to design, implement, test, and deploy our project, and needed to utilize a raspberry pi in the process. Our team also primarily had experience writing backend applications in python and simple frontend applications with javascript, HTML, and CSS. As a result, we determined the most effective way for us to build this application was to create a python-based REST API [14] that could be hosted on a raspberry pi, and would have a simple web frontend that would make requests to the backend to view, edit, and manage tasks.



\chapter{Ethical Considerations}
Students have a wide variety of backgrounds and needs when it comes to creating and managing tasks. Prior work emphasizes that digital learning and productivity tools must take into account diverse accessibility requirements to avoid excluding certain groups of learners \citep{seale2013learning}. We wanted to be mindful of this when designing our website by ensuring that it is accessible for a wide variety of people. This meant providing features like the ability to change text to more dyslexic-friendly fonts or to be larger, ensuring there was enough space for users to enter tasks in different languages \citep{al2016universal}, and providing options for how much information a user wants to see at once in case they may feel overwhelmed seeing too many tasks at once \citep{spina2019wcag}. International perspectives on accessibility stress that tools should be adaptable to diverse cultural and linguistic contexts \citep{world2011world}.

We also wanted to ensure our app was available to as many students as possible, as one of the primary issues with existing tools is that their cost or functionality made them inaccessible to many students. Studies have noted that cost and resource constraints are significant barriers to equitable access in higher education technology \citep{selwyn2021education}. This is why we made our web app freely available and able to be hosted using freely available tools. We also ensured the app was efficient and could run on low-end computer hardware, which aligns with research emphasizing the importance of universal design and low-barrier entry points \citep{rose2002teaching}. Other work on inclusive HCI methods highlights the importance of testing accessibility features systematically \citep{lazar2017research}.


One of our last big accessibility features was to make the tool more accessible by allowing users to add tasks via natural language.  Natural language interfaces have been shown to improve usability for users with varying levels of technical proficiency \citep{shneiderman2010designing}, and they can make interaction with digital tools more intuitive for students unfamiliar with structured task management systems. Similar findings are reported in educational technology studies, where natural language input has been linked to improved engagement \citep{junco2012relationship}.


\chapter{Implementation}
We began with a list of base requirements that our tool needed to be able to handle. The tool needed to allow for users to create accounts to store their task data. Both the frontend and backend needed to be able to handle logins, sessions, and invalid login attempts. Then, the frontend needed to be able to show a list of tasks and give the user the ability to add, edit, delete or mark tasks as complete. Likewise, the backend needed to have endpoints to enable each of those functions for a specified user for a specified task. These functions needed to be clear and doable from both desktop and mobile browsers. Additionally, we wanted to add “chatbot” functionality to add tasks via natural language. On the frontend, this meant adding an interface that allowed users to type in tasks they wanted to add with the details they cared about. On the backend, we created an additional endpoint to parse the user’s request using a structured format, and then making additional requests to the relevant endpoints. The backend also needed to handle requests that weren’t structured properly, and respond with appropriate messages to explain to the user how to make a correct request. 

Once we finished developing the frontend and backend, we needed to test the application to ensure it ran properly on the Raspberry Pi, which was our baseline for hardware that the app could run on. The Raspberry Pi has frequently been used in educational and lightweight computing projects due to its accessibility and low cost \citep{richardson2014getting}. It needed to be able to install the frameworks we needed, clone the repository, and start up the web app to be accessible from the internet. For the backend, we selected Python with a REST API architecture, leveraging lightweight frameworks such as Flask that are commonly used in educational and rapid development contexts \citep{grinberg2018flask}. We then ran tests to ensure everything worked as expected.


\chapter{Reflection}
Our team has gained hands-on experience with both the technical and organizational aspects of developing a web-based application when working with this project. One key takeaway was the importance of matching the limited timeframe and resources available. We also recognized trying to include too many features would have compromised stability. From this course project, our team also learned how essential accessibility considerations are in making sure that a tool can be used by students. Finally, this project highlighted the value of collaboration within team members: dividing responsibilities, and communicating effectively. 


\chapter{Contributions}
Yoav and Ha co-wrote the abstract, introduction, background, methods, and ethical considerations. Methods were compiled based on notes written by each team member. Ha compiled and structured the references. Yasiru and Adnan wrote the implementation section.


\bibliographystyle{plainnat}
\bibliography{references}


\end{document}

%%% Local Variables: ***
%%% mode: latex ***
%%% TeX-master: "main.tex"  ***
%%% ispell-local-dictionary: "british"  ***
%%% End: ***